\chapter{Let's get started}
Here I will illustrate the basics of a RICE. What it means and what are the various components, as well as what they do and why they are there

\section{Before we begin}
I consider part of the customization not just the aesthetic, but also the functionality of the system, so here there will be more than just theme customization, but some light scripting as well.

\section{What makes a system}
You need a lot of parts to make the system actually work. I will be assuming that you have the vanilla Linux Kernel with default parameters, systemd as your init, and GRUB as your bootloader. After that, you will need :
\begin{itemize}
    \item An Operating System
    \item A Display Server
    \item A Window Manager or a Desktop Enviroment
    \item A panel or a bar
    \item An application launcher
    \item A terminal emulator
    \item A shell
    \item A file manager
    \item A notification Daemon
    \item A compositor, or something to display visual effects
    \item A wallpaper
    \item A color scheme
    \item Some fonts
    \item An audio system
    \item A music player
    \item A system monitor
    \item A lock screen
    \item A Power menu
    \item Screenshots capabilities
    \item Clipboard manager
    \item A text editor
    \item Dotfile management
    \item Miscellaneous cool stuff
    \item A web browser
    \item Themes for every application
    \item Automation for basic system functionalities
    \item Session management
\end{itemize}

Remember that all that you need to use a Linux based OS is electricity, and even that is optional, so if you feel like any of these components are not needed, feel free to non include them in your RICE. With all of that said, let's actually start on something.

\subsection{The Operating System}
Whatever distro you chose to customize, it all boils down to the package manager of choice. As such I will be making every step with Arch Linux in mind, but it's easy to check if the thing I'm talking about is compatible with your distro, and if it is, install it with your package manager(or build from source), and follow the guide like nothing happened.
\vspace{\baselineskip}

The OS is made by 8 main components:
\begin{itemize}
    \item Kernel - This is the black magic that translates whatever you do to the hardware into a language that your computer can understand, so it can follow your commands. Without the kernel, the pc is just a bunch of expensive metal.
    \item Init System - The init "wakes up" every program on your system and keeps them running.
    \item Display Server - This guy makes visual output possible, making the connection between the running programs and your display
    \item Shell - This bad boy interprets commands and tells them to the computer, allowing you to do stuff. Even if you never use the terminal, it's always running
    \item File System - This is the structure of your storage drive. Without it, you cannot save of load anything
    \item System Libraries - These are the computer's tools. They are shared pieces of code that every program can call and use to do their jobs.
    \item Basic System Utilities - These are your tools. You use them to interact with your files and with the system
\end{itemize}
\vspace{\baselineskip}

If you have yet to choose a distro, do not be afraid, for I shall give you guidance. It's actually really easy. If you care about stability above all, go with something Debian based, like Ubuntu or Mint. If you want maximum customization capabilities and bleeding edge updates, go with Arch or any derivative distro. If you still want stability but you do not want to get packages 4 years later than everybody else, go with Fedora. I say Fedora and not any other RedHat distro because I do not enjoy the RedHat logo.
\vspace{\baselineskip}

Usually, everything in the list above comes preinstalled in your distro of choice. To customize everything down to the last bit you can either find a distro that has everything you want preinstalled, install Arch Linux manually and choose what you want, or install Gentoo and compile everything from source for hyperoptimization.
You can always install some components on top of other components, even if it's not recommended. It's always a good idea to have a reliable safety net that you know works, so that ig you break something, you can use your default build to fix your WIP RICE.
To keep it simple, we will focus on choosing the display server, the shell, and everything after that. Operating Systems are a complicated thing and the less you touch them the safer it is, until you know what you're actually doing.

\subsection{The display server}
The most common display servers are:
\begin{itemize}
    \item Wayland - The rising star of the new age of Linux Desktop. It's the default DS on Fedora, Ubuntu, and DEs like GNOME, and even KDE Plasma has a Wayland version. It offers better security, performance and supports more modern features like HDR and mixed DPI displays.
    \item X11 - The traditional Linux display server, that was the standard for decades. It's still widely used and it's a reliable option to fall back on if Wayland breaks. It's getting replaced by Wayland for the aforementioned reasons, but it's still a great option and it also offers the most compatibility
    \item WDDM - The proprietary Windows DS. It has been built into Windows since Vista and it handles every graphic operation in Windows
    \item Quartz Compositor - Apple's proprietary DS, it's built into macOS and it does the same job as every display server ever.
\end{itemize}
\vspace{\baselineskip}

Our choices are between Wayland and X11, since we are using Linux. I personally use Wayland, and that's what I recommend.

\subsection{The window manager}
This is the computer's interior designer. It decides where the windows appear on screen, how they look, what happens when you drag, resize or close them, and if the windows stack automatically or not.

\subsection{The bar}
This is your car's dashboard, it tells you the time, your battery level, and more information

\subsection{The application launcher}
This is pretty self-explainatory, it launches applications for you instead of having to dig through your folders to find the program you need

\subsection{The terminal emulator}
This is the part where you can feel like a \emph{supaa hackaa}, by typing commands in text form instead of clicking buttons with your mouse

\subsection{The shell}
This is the language of the terminal, and also the translator

\subsection{The file manager}
This is the file cabinet of the computer. It allows you to navigate files, open them, delete them, create more, and so on

\subsection{The notification deamon}
This sends you notifications when stuff happens, like when your music player skips a song, or if your battery is dying

\subsection{The compositor}
Now we get into prime RICE territory, this is the program responsible for all of the cool effects on your desktop, like shadows, blur, transparency, and animations

\subsection{The wallaper}
If you do not know what a wallpaper is, maybe you should consider a career on an oil rig and never touch a computer again

\subsection{The color scheme}
This is important if you want to have a cohesive theme all around your pc. You can go for a monochrome setup, or a really discordant palette with crazy colors and bright accents

\subsection{The fonts}
Please tell me that you know what a font is

\subsection{The audio system}
This controls everything that relates to audio. Volume levels, which program makes sound, where is outputted, where is inputted and so on

\subsection{The music player}
Everybody listens to music, and you need a way to play it

\subsection{System monitor}
This is useful for diagnostics. Which process is consuming the most resources, at which temperature a said component is running

\subsection{The lockscreen}
You need this if you want to walk away from your computer and not have anybody mess with it while you're gone

\subsection{The power menu}
This allows you to exit your computer in various ways without using the scary terminal. Shutdown, Reboot, Logout, and Sleep are the main modes

\subsection{Screenshots}
You need a way to take screenshots

\subsection{Clipboard manager}
This remembers what you copy. Not just the last thing, everything you CTRL+C stays here

\subsection{Configuration Files}
This is what you're here for. The files that tell the computer what to do and what dress does in need to wear while doing it

\subsection{Dotfile Managment}
You may fuck something up real bad while learning to rice. It is a good idea to use a backup system so you can return to an old version of your files without needing to rewrite everything. Just use git

\subsection{Miscellaneous cool stuff}
Its not a true RICE if you do not have random stuff that does nothing but look cool

\subsection{Browser}
You gotta go on the internet to google everything you do not know or remember

\subsection{Application Theming}
Isn't it just the worst when you have your beautiful system with the perfect colorscheme, but every program you open has its own? Well, change that

\subsection{Scripts and Automation}
It is very useful to add custom scripts to your pc so that it better suits your needs

\subsection{Login Manager}
Also known as the greeter screen, it allows you to login as a user with your password and which desktop do you want to use

\section{Going Forward}
Now that we've extablished every part of the system and what does it do, it's time to do some customization

