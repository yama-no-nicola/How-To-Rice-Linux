\chapter{Let's get started}
Here I will illustrate the basics of a RICE. What it means and what are the various components, as well as what they do and why they are there

\section{Before we begin}
I consider part of the customization not just the aesthetic, but also the functionality of the system, so here there will be more than just theme customization, but some light scripting as well.

\section{What makes a system}
You need a lot of parts to make the system actually work. I will be assuming that you have the vanilla Linux Kernel with default parameters, systemd as your init, and GRUB as your bootloader. After that, you will need :
\begin{itemize}
    \item An Operating System
    \item A Display Server
    \item A Window Manager or a Desktop Enviroment
    \item A panel or a bar
    \item An application launcher
    \item A terminal emulator
    \item A shell
    \item A file manager
    \item A notification Daemon
    \item A compositor, or something to display visual effects
    \item A wallpaper
    \item A color scheme
    \item Some fonts
    \item An audio system
    \item A music player
    \item A system monitor
    \item A lock screen
    \item A Power menu
    \item Screenshots capabilities
    \item Clipboard manager
    \item A text editor
    \item Dotfile management
    \item Miscellaneous cool stuff
    \item A web browser
    \item Themes for every application
    \item Automation for basic system functionalities
    \item Session management
\end{itemize}

Remember that all that you need to use a Linux based OS is electricity, and even that is optional, so if you feel like any of these components are not needed, feel free to non include them in your RICE. With all of that said, let's actually start on something.

\subsection{The Operating System}
Whatever distro you chose to customize, it all boils down to the package manager of choice. As such I will be making every step with Arch Linux in mind, but it's easy to check if the thing I'm talking about is compatible with your distro, and if it is, install it with your package manager(or build from source), and follow the guide like nothing happened.
\pagebreak

If you have yet to choose a distro, do not be afraid, for I shall give you guidance. It's actually really easy. If you care about stability above all, go with something Debian based, like Ubuntu or Mint. If you want maximum customization capabilities and bleeding edge updates, go with Arch or any derivative distro. If you still want stability but you do not want to get packages 4 years later than everybody else, go with Fedora. I say Fedora and not any other RedHat distro because I do not enjoy the RedHat logo.

\subsection{The display server}
